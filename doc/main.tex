\documentclass{bioinfo}
\copyrightyear{2015} \pubyear{2015}

\access{Advance Access Publication Date: Day Month Year}
\appnotes{Application Note}

\usepackage{url}

\begin{document}
\firstpage{1}

\subtitle{Application Note}

\title[Indel Historian: phylogenetic ancestral reconstruction]{Indel Historian: phylogenetic ancestral reconstruction}
\author[Ian Holmes]{Ian Holmes$^{\text{\sfb 1}*}$}
\address{$^{\text{\sf 1}}$Department of Bioengineering, University of California, Berkeley, 94703, USA.}

\corresp{$^\ast$To whom correspondence should be addressed.}

\history{Received on XXXXX } %; revised on XXXXX; accepted on XXXXX}

\editor{Associate Editor: XXXXXXX}

\abstract{
{\bf Motivation.}
{\bf Results.}
{\bf Availability and Implementation.}
Indel Historian is available at \url{https://github.com/ihh/indelhistorian} under the Creative Commons Attribution 3.0 US license.
It is written in C++11.
{\bf Contact.}
Ian Holmes {\tt ihholmes+indelhistorian@gmail.com}.
{\bf Supplementary Information.}
None.
}

\maketitle

\section{Introduction}

Multiple sequence alignments are used for different purposes in bioinformatics,
and these different applications have led to different criteria for evaluating
the accuracy of multiple alignment tools

Homology modeling in protein structure prediction
leads to accuracy metrics that quantify the number of correctly aligned residues
compared to a benchmark dataset of structurally-informed reference alignments

Significant progress has been made optimizing these metrics

Successful techniques include using Bayesian decision theory to maximize the expected
score under the metrics \cite{TCOFFEE,ProbCons,AMAP,FSA}
and fine-tuning the alignment scoring function to reflect known details of
protein sequences under selection for folding, such as the reduced likelihood of indels in
hydrophobic regions \cite{MAFFT,MUSCLE}

These have been accompanied by performance optimizations at the rate-limiting steps
of multiple alignment, including all-versus-all sequence comparison \cite{FSA,MUSCLE}


The study of sequence evolution and molecular phylogenetics gives rise to
an alternative set of criteria for measuring alignment accuracy

Previous work showed that, for the purposes of estimating
molecular evolutionary parameters
such as the rates of insertion-deletion events \cite{Westesson2012-zg}
or the ratio between synonymous abd nonsynonymous substitution rates \cite{Redelings},
it is best to use an explicit probabilistic model of the sequence evolution process
and to treat sequence alignment rigorously as a missing data estimation problem
under this model


The ProtPal program is one such approach which uses the theory of
weighted finite-state transducers
These are automata which can be multiplied like substitution matrices
and so can be thought of as a strict generalization of Felsenstein's likelihood-phylogenetic
approach to multiple sequence alignment
(in fact, the use of transducers makes Felsenstein's pruning algorithm formally
equivalent to a special case of Sankoff's multiple alignment algorithm)

A previously published benchmark of alignment methods on simulated sequence data
showed that most alignment algorithms introduce systematic biases into the
estimation of indel rates, and that these biases are typically substantially worse
at higher indel rates.

ProtPal was found to be the most accurate algorithm for indel rate estimation,
followed by PRANK and then MUSCLE.

However, the implementation of ProtPal published with that benchmark was too slow
for practical use, and no benchmark of ProtPal on structural alignment benchmarks was available

Here, we present a clean implementation of the algorithm underlying ProtPal
in a new tool, Indel Historian
The tool is targeted to evolutionary applications such
as ancestral sequence reconstruction and molecular phylogenetics

We also report the results of benchmarking ProtPal on the BAliBase 3.0 benchmark

\begin{methods}
\section{Methods}

\cite{Westesson2012-zg}

\end{methods}



Comparison to MCMC tools
BaliPhy, PhylogenyCafe, HandAlign


\section{Results}



\section{Discussion}



The best-performing method MUSCLE introduces heuristics into its scoring scheme
that improve accuracy.

It is harder to incorporate these sorts of modifications into a rigorous phylogenetic model
primarily because they violate the assumption of independence
between indel and substitution processes
that most such models rely on

Decision theory may also be useful in conjunction with MCMC tools \cite{HermanHeinEtc}



\section{Availability}

Indel Historian is available at \url{https://github.com/ihh/indelhistorian} under the Creative Commons Attribution 3.0 US license. It is written in C++11, and compiles on a POSIX system with Clang (v.6.1.0). It requires the GNU Scientific Software library (libgsl), the ZLib compression library (libz), and the Boost C++ library.

\section*{Acknowledgements}

Indel Historian includes code from Ivan Vashchaev (gason), Heng Li (kseq), Rafael Baptista (stacktrace), David Robert Nadeau (memsize), Roger Pate (string escaping), Konrad Rudolph (index sort), and the StackOverflow community.

\section*{Funding}

This work has been supported by NHGRI grant R01-HG004483.

\bibliographystyle{natbib}
%\bibliographystyle{bioinformatics}
%\bibliographystyle{achemnat}
%\bibliographystyle{plainnat}
%\bibliographystyle{abbrv}
%
%\bibliographystyle{plain}
%
%\bibliography{Document}


\bibliography{references}
\end{document}
